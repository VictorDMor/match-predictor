% -----------------------------------------------------------------------------
%   Arquivo: ./01-texto/desenvolvimento.tex
% -----------------------------------------------------------------------------


\section{Desenvolvimento}\label{sec:desenvolvimento}

Antes de se desenvolver o código da solução para o problema proposto, é preciso entender os desafios que acompanham este problema. São eles: 
\noindent\textbf{(1)} Encontrar um volume de dados suficiente para o aprendizado 
\noindent\textbf{(2)} Como transformar e modelar esses dados de forma a se transformar numa entrada plausível para o modelo matemático? 
\noindent\textbf{(3)} Qual o melhor algoritmo de aprendizado para a predição?

\subsection{Primeiro desafio}

Para o primeiro desafio, é preciso frisar que todos os tipos de problemas que são resolvidos com aprendizado de máquina supervisionado devem ter um conjunto de dados de treinamento de tamanho satisfatório com o objetivo de cobrir todas as possibilidades e saídas, assim permitindo que o algoritmo 'conheça' todo o escopo do problema e possa gerar saídas com grande nível de precisão.

No tocante à predição de resultados de partidas de futebol isso não é diferente. É necessário uma grande base de dados que contenha o histórico de milhares de partidas de futebol de  várias temporadas. Por isso, foram estudados diversos conjuntos de dados e foi selecionado o conjunto de dados chamado 'European Soccer Database', que contém uma amostra de 25 mil partidas da principal liga de onze países da Europa entre os anos de 2008 e 2016.

Mais detalhadamente falando, essa base de dados possui informações sobre jogos de futebol, jogadores, equipes, países, campeonatos e habilidades técnicas de jogadores. Trata-se de um conjunto de dados bastante completo e apto a ser usado para a modelagem do problema proposto.

\subsection{Segundo desafio}

Com a base de dados em mãos, foi necessário um estudo sobre como preparar os dados de entrada das amostras para servirem de dados de entrada para o algoritmo de aprendizado. A primeira decisão tomada foi a de não usar 100\% dessas amostras, uma vez que em cada um dos onze países contidos na base de dados as influênc\subsection{Próximos passos}ias no resultado de um jogo e até mesmo a maneira de praticar o esporte é diferente. Ou seja, há variáveis que influenciam mais do que outras no resultado de uma partida em um determinado campeonato, e essas variáveis podem influenciar menos em detrimento de outras em outra liga de outro país.

Por isso, decidiu-se por usar as amostras apenas de uma liga em particular, a 'Premier League', a primeira divisão do campeonato inglês de futebol. Essa liga foi escolhida por além de ser uma das mais famosas do mundo, é a liga que tem o nível de detalhe mais rico no que diz respeito à qualidade dos dados extraídos. Ademais, a 'Premier League' é a liga que mais movimenta dinheiro no mundo, dando um total de 400 milhões de libras esterlinas, o equivalente a 1,7 bilhões de reais em premiação para as equipes, além de ser a liga que mais movimenta dinheiro em direitos de transmissão e transferência de jogadores.

Finalmente, foram escolhidas para treinamento um total de 3.040 amostras, ou seja, 3.040 jogos de futebol do campeonato inglês entre os anos de 2008 e 2016. Após, foi preciso uma nova análise desses dados para decidir como o modelo matemático de previsão do resultado das partidas seria construído.

Para a previsão em si, um sistema de influências foi criado. Para efeitos de simplificação, foram escolhidas apenas quatro fatores que podem influenciar o resultado de um jogo:

\textbf{(1)} Mando de campo

\textbf{(2)} Pontuação da equipe no campeonato desde o início até a rodada a qual a partida será realizada

\textbf{(3)} Pontuação da equipe no campeonato nos últimos sete jogos

\textbf{(4)} Média quantificada das habilidades do time titular de cada equipe


É importante mencionar que todas os fatores têm a mesma importância, ou seja, implicitamente falando todos os fatores tem peso 1.

Decidiu-se em cada fator comparar as equipes entre si e, após a comparação, a cada fator é atribuido um valor, avaliado em:

\textbf{(a)} 1, em caso de vitória do mandante no fator

\textbf{(b)} -1, em caso de vitória do visitante no fator

\textbf{(c)} 0, em caso de empate no fator


Assim, o resultado da partida é previsto após a soma dos valores de cada fator, semelhante às mesmas regras listadas acima, caso a soma dos valores seja

\textbf{(a)} maior ou igual a 1, vitória do time mandante

\textbf{(b)} menor ou igual a -1, vitória do time visitante

\textbf{(c)} 0, empate

\subsection{Construção do algoritmo}
O objetivo final é que o algoritmo receba uma partida do campeonato inglês como parâmetro de entrada, este parâmetro é subdividido em três parâmetros: 'nome do time da casa', 'nome do time visitante', 'ano em que a partida é realizada'.

Com isso, o algoritmo busca a rodada da liga pertence esta partida, além de calcular dados pertinentes a cada uma das equipes, como pontuação, jogadores que iniciarão aquela partida como titulares, número de gols marcados, número de gols sofridos e a média quantificada da habilidade técnica desses jogadores titulares.

Após a extração desses dados, estes são colocados em análise com o objetivo de se calcular os fatores listados na subseção acima e, assim, gerar a previsão do resultado do jogo.

Tecnicamente falando, optou-se por usar a linguagem de programação Python com a biblioteca Pandas para o tratamento dos dados e a comunicação com a base de dados, que foi criada usando a linguagem de banco de dados SQL e permite visualização pelo script Python através da biblioteca SQLite.

O script criado, chamado de Analyst, recebe por parâmetro em linha de comando os dados descritos acima, sua execução é feita por:

{\tiny
	python analyst.py <mandante> <visitante> <ano>
}

Quando executado, o script se conecta com a base de dados pela biblioteca SQLite, extrai os dados da base de dados tomando como referência as informações obtidas por linha de comando usando o Pandas, os converte em tipos de variável primários, faz os cálculos pertinentes explicados anteriormente e retorna para o usuário uma mensagem indicando qual será o resultado da partida (em inglês). 

\subsection{Treinamento e precisão da predição}

Após prever o resultado da partida, o algoritmo reexecuta todos os passos N vezes, sendo N o tamanho da amostra, ou seja, agora o modelo é executado para todas as partidas da amostra. Porém, dessa vez o resultado da previsão é comparado com o resultado real da partida (1 para vitória do mandante, -1 para vitória do visitante e 0 para o empate) e, em caso de igualdade, é contabilizado um acerto para o algoritmo, caso contrário, é contabilizado um erro para o algoritmo. A precisão é calculada como a soma dos acertos dividida pelo número de amostras.

\subsection{Resultados e terceiro desafio}

Para o caso explanado acima, onde todos os fatores/variáveis têm a mesma influência sob o resultado final, sem quaisquer tipo de aprendizado supervisionado, foi obtida uma precisão na ordem de 42,23\% sob as 3.040 amostras comparadas. Portanto, é fortemente recomendado o uso de um algoritmo de aprendizado de máquina supervisionado onde o desafio final torna-se ajustar as influências (ou pesos) de cada variável para que haja um aumento no número de acertos da solução proposta e, consequentemente, um aumento na precisão. Afinal de contas, trata-se de um espelho da realidade, pois de fato alguns fatores são mais preponderantes no resultado final de uma partida do que outros, por isso é fundamental o uso de aprendizado de máquina, ou \textit{Machine Learning}.

Na visão de aprendizado de máquina, conclui-se que prever os resultados de uma partida de futebol trata-se de um problema de classificação, onde as partidas devem ser classificadas em vitória do mandante, vitória do visitante ou empate. Por isso, três algoritmos são propostos para treinar os pesos:
\\

\textbf{(1)} Redes Neurais Recorrentes
\textbf{(2)} Supporting Vector Machines (SVM)
\textbf{(3)} Regressão Logística

A escolha do melhor algoritmo a ser usado depende diretamente de como os dados são modelados. Algumas aplicações interessantes podem ser utilizadas a partir desse modelo. Como por exemplo a otimização de esforços em determinadas áreas por parte de um time de futebol. Por exemplo, se em um país A o modelo após treinamento detectou uma influência maior do mando de campo nas vitórias do time da casa, é possível concentrar recursos na melhoria do estádio desse time. Ou em um país B onde a eficácia defensiva é fundamental nos resultados positivos de uma equipe, é possível criar uma formação tática ou estratégia que tenha ênfase na defesa.

\subsection{Próximos passos}

Além da efetiva implementação de um algoritmo de treinamento no modelo proposto, várias outras melhorias podem ser efetuadas visando um aumento na precisão da solução, como por exemplo:

\textbf{(1)} Uso de mais fatores, como entrosamento da equipe
\textbf{(2)} Melhor modelagem dos atributos dos jogadores
\textbf{(3)} Análise de sensibilidade usando sistemas Fuzzy

Tudo isso, combinado, traria importante crescimento no desenvolvimento do modelo de negócio e, após, em resultados positivos.

\subsection{Conclusão}

Prever resultados de futebol é só uma simples aplicação de métodos matemáticos e aprendizado de máquina e que torna fácil o entendimento da teoria que cerca esses dois assuntos. É impossível alcançar a perfeição nas predições, uma vez que o fator humano envolvido nos resultados das partidas é ainda impossível de se modelar. Além disso, um algoritmo hipotético que acertasse 100\% das previsões tiraria aquela que é a motivação em se acompanhar esportes, que é a emoção envolvida acerca da confiança em um resultado positivo da sua equipe que o torcedor.

Sendo o fator humano tão importante até mesmo para a rentabilidade do futebol no meio profissional, existem certas fronteiras as quais, por uma questão ética, a Inteligência Artificial deve ser impedida de alcançar a perfeição, e prever resultados do esporte mais amado do planeta com 100\% de acertos é certamente uma dessas fronteiras.

